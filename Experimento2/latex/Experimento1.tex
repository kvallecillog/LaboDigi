%%%%%%%%%%%%%%%%%%%%%%%%%%%%%%%%%%%%%%%%%%%%%%%%%%%%%%%%%%%%%%%%%%%%%%%%%%%%%%%
%%%%%%%%%%%%%%%%%%%%%%%%%%%%%%%%%%%%%%%%%%%%%%%%%%%%%%%%%%%%%%%%%%%%%%%%%%%%%%%
%%%%%%%%%%%%%%%%%%%%%%%%%%%%%%%%%%%%%%%%%%%%%%%%%%%%%%%%%%%%%%%%%%%%%%%%%%%%%%%
\documentclass[12pt,letterpaper]{article}     % Tipo de documento y otras especificaciones
\usepackage[utf8]{inputenc}                   % Para escribir tildes y eñes
\usepackage[spanish]{babel}                   % Para que los títulos de figuras, tablas y otros estén en español
\addto\captionsspanish{\renewcommand{\tablename}{Tabla}}					% Cambiar nombre a tablas
\addto\captionsspanish{\renewcommand{\listtablename}{Índice de tablas}}		% Cambiar nombre a lista de tablas
\usepackage{geometry}                         
\geometry{left=18mm,right=18mm,top=21mm,bottom=21mm} % Tamaño del área de escritura de la página
\usepackage{ucs}
\usepackage{amsmath}      % Los paquetes ams son desarrollados por la American Mathematical Society
\usepackage{amsfonts}     % y mejoran la escritura de fórmulas y símbolos matemáticos.
\usepackage{amssymb}
\usepackage{graphicx}     % Para insertar gráficas
\usepackage{epstopdf}     % Permite el uso de eps
\usepackage[lofdepth,lotdepth]{subfig}	% Para colocar varias figuras
\usepackage{unitsdef}	  % Para la presentación correcta de unidades
\usepackage{pdfpages}   %incluir paginas de pdf externo, para los anexos
\usepackage{appendix}   %para los anexos
\renewcommand{\unitvaluesep}{\hspace*{4pt}}	% Redimensionamiento del espacio entre magnitud y unidad
\usepackage[colorlinks=true,urlcolor=blue,linkcolor=black,citecolor=black]{hyperref}     % Para insertar hipervínculos y marcadores
\usepackage{float}		% Para ubicar las tablas y figuras justo después del texto
\usepackage{booktabs}	% Para hacer tablas más estilizadas
\batchmode
%\usepackage{apacite}
\bibliographystyle{plain} 
\pagestyle{plain} 
\usepackage{ltablex}
\pagenumbering{arabic}
\usepackage{lastpage}
\usepackage{fancyhdr}	% Para manejar los encabezados y pies de página
\pagestyle{fancy}		% Contenido de los encabezados y pies de pagina
\usepackage{multicol} 
%%%%%%%%%%%%%%%%%%%%%%%%%%%%%%%%%%%%%%%%%%%%%%%%%%%%%%%%%%%%%%%%%%%%%%%%%%%%%%%
%No modificar las líneas anteriores
%%%%%%%%%%%%%%%%%%%%%%%%%%%%%%%%%%%%%%%%%%%%%%%%%%%%%%%%%%%%%%%%%%%%%%%%%%%%%%%
\lhead{IE-0424 LABORATORIO DE CIRCUITOS DIGITALES I}
\chead{}
\rhead{Experimento 1}	% Aquí va el numero de experimento, al igual que en el titulo
\lfoot{Escuela de Ingeniería Eléctrica}
\cfoot{\thepage\ de \pageref{LastPage}}
\rfoot{Universidad de Costa Rica}
%%%%%%%%%%%%%%%%%%%%%%%%%%%%%%%%%%%%%%%%%%%%%%%%%%%%%%%%%%%%%%%%%%%%%%%%%%%%%%%
\author{Natalia Araya Campos, B00448 \\ Alejandro Leon Torres, B13645 \\ Andres Mora Zuñiga, B14463 \\ Kenneth Vallecillo González, B16750  \\ {\small Grupo 1}\\ \\ \\ \\Profesor: Diego Valverde G\vspace*{2.0in}}
\title{Universidad de Costa Rica\\{\small Facultad de Ingeniería\\Escuela de Ingeniería Eléctrica\\IE-0424 LABORATORIO DE CIRCUITOS DIGITALES I \\II ciclo 2014\\\vspace*{0.55in} Experimento I}\\  MiniAlu Xilinx\\ \vspace*{1.35in}}
%\date{}  		
%%%%%%%%%%%%%%%%%%%%%%%%%%%%%%%%%%%%%%%%%%%%%%%%%%%%%%%%%%%%%%%%%%%%%%%%%%%%%%%
\begin{document}	% Inicio del documento
%%%%%%%%%%%%%%%%%%%%%%%%%%%%%%%%%%%%%%%%%%%%%%%%%%%%%%%%%%%%%%%%%%%%%%%%%%%%%%%
\pdfbookmark[1]{Portada}{portada} 	% Marcador para el título

\maketitle							% Título

\thispagestyle{empty} 

\newpage
\tableofcontents
\listoffigures
\newpage
\setcounter{secnumdepth}{4}
%%%%%%%%%%%%%%%%%%%%%%%%%%%%%%%%%%%%%%%%%%%%%%%%%%%%%%%%%%%%%%%%%%%%%%%%%%%%%%%
\section{Ejercicio I}
Para implementar la multiplicación se debe agregar en el archivo \textit{Defintions.v} la siguiente línea:
\begin{verbatim}
`define MUL   4'b1000
\end{verbatim}
Y en el archivo \textit{MiniAlu.v} se debe agregar y alterar el código, se muestra a continuación los cambios más significativos en el archivo:
\begin{verbatim}
// Se cambia de reg a 
// reg wired como lo indica el enunciado.
reg signed [15:0] rResult;

// Se crea una variable temporal para almacenar el resultado de la
// multiplicación y después tomar la parte más significativa y ponerla
// en rResult.
reg signed [31:0] rResultTemp;


// Se cambia el tipo de wire en un wired signed como lo indica el enunciado
// esto con el fin de obtener un resultado correcto en la multiplicación SMUL.
wire signed [15:0] wSourceData0_signed,wSourceData1_signed;



// Se asigna el valor de las sources a las sources con signo
// Esto permite tomar los valores de las sources
// y darles signo para utilizarlos en la multiplicación SMUL
// y utilizar sus valores sin signo para las demás operaciones.
assign wSourceData0_signed = wSourceData0;
assign wSourceData1_signed = wSourceData1;




//-------------------------------------
`MUL:
begin
	rFFLedEN     <= 1'b0;
	rBranchTaken <= 1'b0;
	rWriteEnable <= 1'b1;
	rResultTemp  <= wSourceData0_signed * wSourceData1_signed;
	rResult		 <= rResultTemp [31:16];
end	
//-------------------------------------
\end{verbatim}



\section{Ejercicio II}
Se debe agregar el módulo \textit{imul} el cual se muestra a continuación:
\begin{verbatim}

\end{verbatim}
%%%%%%%%%%%%%%%%%%%%%%%%%%%%%%%%%%%%%%%%%%%%%%%%%%%%%%%%%%%%%%%%%%%%%%%%%%%%%%%
\end{document}
%%%%%%%%%%%%%%%%%%%%%%%%%%%%%%%%%%%%%%%%%%%%%%%%%%%%%%%%%%%%%%%%%%%%%%%%%%%%%%%
%%%%%%%%%%%%%%%%%%%%%%%%%%%%%%%%%%%%%%%%%%%%%%%%%%%%%%%%%%%%%%%%%%%%%%%%%%%%%%%
%%%%%%%%%%%%%%%%%%%%%%%%%%%%%%%%%%%%%%%%%%%%%%%%%%%%%%%%%%%%%%%%%%%%%%%%%%%%%%%